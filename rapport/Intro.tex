\section{Introduction}

\subsection{Contexte}

Dans le cadre de notre projet, nous avons développé une application de suivi et d'analyse des données GPS pour les cyclistes. L'objectif principal de ce projet est de fournir aux utilisateurs des informations précises et utiles sur leurs trajets, leur performance et leur sécurité.

\subsection{Objectifs}

Les objectifs de ce projet sont multiples :
\begin{itemize}
    \item Collecter et analyser les données GPS des trajets effectués par les cyclistes.
    \item Fournir des visualisations claires et intuitives des trajets et des performances.
    \item Permettre aux utilisateurs de filtrer et de personnaliser les données affichées selon leurs besoins.
    \item Améliorer la sécurité des cyclistes en identifiant les zones à risque et en fournissant des recommandations.
\end{itemize}

\subsection{Méthodologie}

Pour atteindre ces objectifs, nous avons suivi une méthodologie structurée en plusieurs étapes :
\begin{enumerate}
    \item \textbf{Collecte des données} : Utilisation de capteurs GPS pour enregistrer les trajets des cyclistes.
    \item \textbf{Analyse des données} : Traitement et analyse des données collectées pour extraire des informations pertinentes.
    \item \textbf{Développement de l'application} : Conception et implémentation d'une application web pour visualiser et interagir avec les données.
    \item \textbf{Tests et validation} : Évaluation de l'application avec des utilisateurs réels pour s'assurer de sa fiabilité et de son utilité.
\end{enumerate}

\subsection{Structure du rapport}

Ce rapport est structuré comme suit :
\begin{itemize}
    \item \textbf{Section 1 : Introduction} - Présentation du contexte, des objectifs et de la méthodologie du projet.
    \item \textbf{Section 2 : Collecte des données} - Description des méthodes et des outils utilisés pour la collecte des données GPS.
    \item \textbf{Section 3 : Analyse des données} - Explication des techniques d'analyse des données et des résultats obtenus.
    \item \textbf{Section 4 : Développement de l'application} - Détails sur la conception et l'implémentation de l'application web.
    \item \textbf{Section 5 : Tests et validation} - Résultats des tests utilisateurs et validation de l'application.
    \item \textbf{Section 6 : Conclusion} - Synthèse des résultats et perspectives pour les travaux futurs.
\end{itemize}

\subsection{Conclusion}

En conclusion, ce projet vise à fournir aux cyclistes un outil puissant pour suivre et analyser leurs trajets, améliorer leur performance et assurer leur sécurité. Le développement de cette application repose sur une méthodologie rigoureuse et des technologies avancées pour offrir une expérience utilisateur optimale.
